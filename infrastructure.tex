
%% Beamer defines a range of its own theorems
\ifx\beamer\undefined

\theoremstyle{plain}
\newtheorem{theorem}{Theorem}[section]
\newtheorem{lemma}[theorem]{Lemma}
\newtheorem{corollary}[theorem]{Corollary}

\theoremstyle{definition}
\newtheorem*{remark}{Remark}
\newtheorem*{notation}{Notation}
\newtheorem{definition}{Definition}
\newtheorem{conjecture}{Conjecture}
\newtheorem{example}{Example}

\else 
  %% Intentionally left blank
\fi

\makeatletter
\newcommand\xlabel[2][]{\phantomsection\def\@currentlabelname{#1}\label{#2}}
\makeatother

\newenvironment{rules}[1][{}]{\begin{mathpar}}{\end{mathpar}}

%% This puts the name on the top
\NewDocumentCommand{\defrule}{o o m m}{
  \inferrule*[lab=\IfNoValueTF{#1}{}{\textsc{[#1]}}]
  { #3 }
  { #4 }
  \IfNoValueTF{#2}{}{\xlabel[#1]{\ifdefined\InApx{apx:}\else\fi#2}}
}


\NewDocumentCommand{\ruleref}{m}{{[\textsc{\nameref{#1}}]}}

\NewDocumentCommand{\stepref}{m m}{\hyperref[#1]{\textsc{[#2]}}}